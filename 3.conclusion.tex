\section{Conclusion}

The problem tackled in this survey was to construct an assignment that was
persistent and balanced, not only
on average but with high probability, thus preventing an adversarial choice of
objects from unbalancing it.
We exposed two different settings in which
this problem naturally emerges: the job assignment in a distributed server and
the simulation of PRAMs on DMMs. In both cases it turned out that if we
employ a randomized hashing scheme it is much more convenient to employ
the two choices scheme, in this way we get an exponential reduction of the
maximum load. Although proving the effectiveness of the gain was not simple
we did not increase the algorithm complexity significantly, then the algorithm
remains practical to implement.

\paragraph{Further Readings.} The reader that is interested in finding
other applications of this paradigm may look at the paper from
Pagh and Rodler \cite{Pagh} that brilliantly exploit the two
choices to build an hash table with wort-case constant lookup time. Another
interesting application is the work of Cole {\em et al.} \cite{Cole} that employs
this technique to minimize congestion in message routing over two-fold butterfly
networks.


%%% Local Variables:
%%% mode: latex
%%% TeX-master: "paper"
%%% End:
