\section{Introduction}

A famous problem in algorithmic literature is the ``Balls and Bins Problem'',
in this survey we will expose two settings in which this problem arises
and present a brilliant approach employing a randomized algorithm to
solve it efficiently.
Suppose that we dispose of $n$ balls and we want to put them into $m$
bins so that each bin gets approximately the same number of balls, to be
more precise we want some guarantee that the maximum number of balls
in any bin does not deviate too much from the average. It is easy to find
a mapping $f:\bigl[1,\dots n]\longrightarrow[1,\dots m]$ that satisfies the
constraints above. Consider now the game in which an adversary want to
find the set $f^{-1}\bigl(\{j\}\bigr)$ for any $j \in [1,\dots m]$ and we
have to prevent him from doing so. Finding a mapping that satisfies also
this last constraint is more complicated,
and we need to employ randomization to do so.
The simplest algorithm to solve this problem\footnote{Exposed by D. E. Knuth in \cite{Knuth}.} consists in selecting a random
hash function $h$ and mapping each ball $i$ to the bin corresponding to
$h\bigl(i\bigr) \pmod{m}$. It turns out that each bin has
less than $O\bigl(\log(n) / \log\bigl(\log(n)\bigr)\bigr)$ balls with high probability\footnote{``With high
  probability'' (w.h.p.) is a widespread formula in
  algorithmic literature and means ``with
  probability at least $1 - 1 / n$''.}. We will see that employing two randomly
chosen hash functions will lower this concentration bound exponentially\footnote{This result is due to Azar
  {\em et al.} \cite{Azar}.}.
In section 2 we will describe this algorithm in details, proving the
concentration bound, and apply it to the problem of balancing a distributed
server back-end. In section 3 we will apply this solution to another interesting
problem in theoretical computer science: the simulation of shared memory
machines (PRAM) on distributed memory machines (DMM), again the two choices approach led to an improvement in the state of
art. 

%%% Local Variables:
%%% mode: latex
%%% TeX-master: "paper"
%%% End:
